\documentclass[$if(fontsize)$$fontsize$,$endif$$if(lang)$$lang$,$endif$$if(papersize)$$papersize$,$endif$$for(classoption)$$classoption$$sep$,$endfor$]{$documentclass$}
$if(fontfamily)$
\usepackage{$fontfamily$}
$else$
\usepackage{lmodern}
$endif$
$if(linestretch)$
\usepackage{setspace}
\setstretch{$linestretch$}
$endif$
\usepackage{amssymb,amsmath}
\usepackage{ifxetex,ifluatex}
\ifnum 0\ifxetex 1\fi\ifluatex 1\fi=0 % if pdftex
  \usepackage[T1]{fontenc}
  \usepackage[utf8]{inputenc}
$if(euro)$
  \usepackage{eurosym}
$endif$
\else % if luatex or xelatex
  \ifxetex
    \usepackage{mathspec}
    \usepackage{xltxtra,xunicode}
  \else
    \usepackage{fontspec}
  \fi
  \defaultfontfeatures{Mapping=tex-text,Scale=MatchLowercase}
  \newcommand{\euro}{€}
$if(mainfont)$
    \setmainfont{$mainfont$}
$endif$
$if(sansfont)$
    \setsansfont{$sansfont$}
$endif$
$if(monofont)$
    \setmonofont[Mapping=tex-ansi]{$monofont$}
$endif$
$if(mathfont)$
    \setmathfont(Digits,Latin,Greek){$mathfont$}
$endif$
\fi
% use upquote if available, for straight quotes in verbatim environments
\IfFileExists{upquote.sty}{\usepackage{upquote}}{}
% use microtype if available
\IfFileExists{microtype.sty}{%
\usepackage{microtype}
\UseMicrotypeSet[protrusion]{basicmath} % disable protrusion for tt fonts
}{}
$if(geometry)$
\usepackage[$for(geometry)$$geometry$$sep$,$endfor$]{geometry}
$endif$
$if(lang)$
\ifxetex
  \usepackage{polyglossia}
  \setmainlanguage{$mainlang$}
  $for(other_langs)$
    \setotherlanguage{$other_langs$}
  $endfor$
\else
  \usepackage[shorthands=off,$lang$]{babel}
\fi
$endif$
$if(natbib)$
\usepackage{natbib}
\bibliographystyle{$if(biblio-style)$$biblio-style$$else$plainnat$endif$}
$endif$
$if(biblatex)$
\usepackage{biblatex}
$if(biblio-files)$
\bibliography{$biblio-files$}
$endif$
$endif$

\usepackage{listings}

$if(lhs)$
\lstnewenvironment{code}{\lstset{language=Haskell,basicstyle=\small\ttfamily}}{}
$endif$
$if(highlighting-macros)$
$highlighting-macros$
$endif$
$if(verbatim-in-note)$
\usepackage{fancyvrb}
\VerbatimFootnotes
$endif$

\usepackage{longtable,booktabs}

\usepackage{graphicx}
\makeatletter
\def\maxwidth{\ifdim\Gin@nat@width>\linewidth\linewidth\else\Gin@nat@width\fi}
\def\maxheight{\ifdim\Gin@nat@height>\textheight\textheight\else\Gin@nat@height\fi}
\makeatother
% Scale images if necessary, so that they will not overflow the page
% margins by default, and it is still possible to overwrite the defaults
% using explicit options in \includegraphics[width, height, ...]{}
\setkeys{Gin}{width=\maxwidth,height=\maxheight,keepaspectratio}

\ifxetex
  \usepackage[setpagesize=false, % page size defined by xetex
              unicode=false, % unicode breaks when used with xetex
              xetex]{hyperref}
\else
  \usepackage[unicode=true]{hyperref}
\fi
\hypersetup{breaklinks=true,
            bookmarks=true,
            pdfauthor={$author-meta$},
            pdftitle={$title-meta$},
            colorlinks=true,
            citecolor=$if(citecolor)$$citecolor$$else$blue$endif$,
            urlcolor=$if(urlcolor)$$urlcolor$$else$blue$endif$,
            linkcolor=$if(linkcolor)$$linkcolor$$else$magenta$endif$,
            pdfborder={0 0 0}}
\urlstyle{same}  % don't use monospace font for urls
$if(links-as-notes)$
% Make links footnotes instead of hotlinks:
\renewcommand{\href}[2]{#2\footnote{\url{#1}}}
$endif$

\usepackage[normalem]{ulem}
% avoid problems with \sout in headers with hyperref:
\pdfstringdefDisableCommands{\renewcommand{\sout}{}}

\setlength{\parindent}{0pt}
\setlength{\parskip}{6pt plus 2pt minus 1pt}
\setlength{\emergencystretch}{3em}  % prevent overfull lines
$if(numbersections)$
\setcounter{secnumdepth}{5}
$else$
\setcounter{secnumdepth}{0}
$endif$
$if(verbatim-in-note)$
\VerbatimFootnotes % allows verbatim text in footnotes
$endif$

% This is a hack from
% <http://tex.stackexchange.com/questions/45105/
% after-notitlepage-my-abstract-lost-formatting
% #answer-45202> when I don't use it one of the abstracts
% has inverted margins. There should be a better solution to
% this problem.
\renewenvironment{abstract}{
    \newpage
    \null\vfil
    \begin{center}%
        \bfseries \abstractname
    \end{center}
}
{\par\vfil\null\clearpage}

$if(definiciones)$
    % Workaround from: <http://tex.stackexchange.com/
    % questions/85696/what-causes-this-strange-interaction-
    % between-glossaries-and-amsmath#answer-85700>
    \makeatletter % undo the wrong changes made by mathspec
    \let\RequirePackage\original@RequirePackage
    \let\usepackage\RequirePackage
    \makeatother
    \usepackage[xindy,section]{glossaries}
    \loadglsentries{$definiciones$}
    \makeglossaries

    % Workaround from: <http://tex.stackexchange.com/
    % questions/109122/latex-glossaries-hyperref-link-
    % only-the-first-occurance-of-an-entry-in-each-sec#answer-283573>
    \renewcommand*{\glslinkcheckfirsthyperhook}{%
        \ifglsused{\glslabel}%
        {%
            \setkeys{glslink}{hyper=false}%
        }%
        {}%
    }
$endif$

\usepackage{etoolbox}
\usepackage{fancyhdr}
% Turn on the style
\pagestyle{fancy}
% Sets both header and footer to nothing
\fancyhf{}
% Set the right side of the footer to be the page number
\fancyfoot[R]{\thepage}
% Patches
\renewcommand{\headrulewidth}{0pt}
\patchcmd{\chapter}{\thispagestyle{plain}}{\thispagestyle{fancy}}{}{}
\patchcmd{\tableofcontents}{\thispagestyle{plain}}{\thispagestyle{fancy}}{}{}
\patchcmd{\listoftables}{\thispagestyle{plain}}{\thispagestyle{fancy}}{}{}
\patchcmd{\listoffigures}{\thispagestyle{plain}}{\thispagestyle{fancy}}{}{}

\usepackage[titletoc]{appendix}
\usepackage{caption}
\usepackage{cleveref}


% Workaround from: <http://tex.stackexchange.com/questions/433/
% vertically-center-text-and-image-in-one-line#answer-17101>
\newcommand{\vcenteredinclude}[1]{\begingroup
\setbox0=\hbox{\includegraphics{#1}}%
\parbox{\wd0}{\box0}\endgroup}

\begin{document}
    \pagenumbering{roman}

    \begin{titlepage}
        \centering
        \bfseries

        \fontsize{18pt}{21.6pt}\selectfont
        \MakeUppercase{$institution$}

        \fontsize{16pt}{19.2pt}\selectfont
        \MakeUppercase{$academic_unit$}

        \fontsize{14pt}{16.8pt}\selectfont
        \MakeUppercase{$city$ - $country$}

        \vspace{1cm}
        \includegraphics[height=33mm]
            {template/Logo_UTFSM_crop.png}

        \vfill

        \fontsize{20pt}{24pt}\selectfont
        \MakeUppercase{$title$}

        \vfill

        \fontsize{14pt}{16.8pt}\selectfont
        \MakeUppercase{$author$}

        \fontsize{12pt}{14.4pt}\selectfont
        \MakeUppercase{
            Memoria de Titulación para optar al título de
            $degree$
        }

        \fontsize{12pt}{14.4pt}\selectfont
        \MakeUppercase{PROFESOR GUÍA: $professor$}

        \fontsize{14pt}{16.8pt}\selectfont
        \MakeUppercase{$month$ - $year$}
        \par

        $if(advertencia)$
            \fontsize{14pt}{16.8pt}\selectfont
            Material de referencia, su uso no involucra
            responsabilidad del autor o de la Institución.
            \par
        $endif$
    \end{titlepage}
    \pagebreak

    $if(agradecimientos)$
        \section*{\centering AGRADECIMIENTOS}
        \input{$agradecimientos$}
        \pagebreak
    $endif$

    $if(dedicatoria)$
        \vspace*{\stretch{2}}
        \begin{flushright}
            \itshape
            \input{$dedicatoria$}
        \end{flushright}
        \vspace{\stretch{1}}
        \pagebreak
    $endif$

    $if(resumen)$
        \begin{abstract}
            \input{$resumen$}
        \end{abstract}
        \pagebreak
    $endif$

    $if(abstract)$
        \begin{english}
            \begin{abstract}
                \input{$abstract$}
            \end{abstract}
        \end{english}
        \pagebreak
    $endif$

    $if(definiciones)$
        \printglossary
        \pagebreak
    $endif$

    $if(toc)$
        {
            \hypersetup{linkcolor=black}
            \setcounter{tocdepth}{$toc-depth$}
            \tableofcontents
        }
    $endif$

    $if(lot)$
        \listoftables
    $endif$

    $if(lof)$
        \listoffigures
    $endif$

    \cleardoublepage
    \pagenumbering{arabic}

    $for(main_content)$
        \input{$main_content$}
        \pagebreak
    $endfor$

    $if(anexos)$
        \begin{appendices}
            $for(anexos)$
                \input{$anexos$}
            $endfor$
        \end{appendices}
    $endif$

    $if(bibliografia)$
        \input{$bibliografia$}
        \pagebreak
    $endif$
\end{document}
